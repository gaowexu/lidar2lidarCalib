\documentclass[
	letterpaper,
	10pt, 
]{CSUniSchoolLabReport}

\addbibresource{sample.bib} 

\usepackage{geometry}
\usepackage{amsmath}
\usepackage{bm}
\usepackage{hyperref}

\geometry{a4paper, top = 2.2cm, bottom = 2.2cm, left=3.2cm, right=3.2cm}



\title{Lidar-Lidar Extrinsic Matrix Manual Calibration User Guide} 

\author{Smart Automobile, Inc.}

\date{\today}

\begin{document}

\maketitle


\section{Notation}

Consider a point in left lidar's frame $P_l = (x_l, y_l, z_l)^T$, its corresponding 3D point in top-center lidar's frame is $P_t = (x_t, y_t, z_t)^T$. The extrinsic matrix from left lidar to top-center lidar's frame is $\bm{E}_{l2t}$. Therefore, we have:
\begin{equation}
P_t = \bm{E}_{l2t} P_l
\end{equation} \\

Consider a point in right lidar's frame $P_r = (x_r, y_r, z_r)^T$, its corresponding 3D point in top-center lidar's frame is $P_t = (x_t, y_t, z_t)^T$. The extrinsic matrix from right lidar to top-center lidar's frame is $\bm{E}_{r2t}$. Therefore, we have:
\begin{equation}
P_t = \bm{E}_{r2t} P_r
\end{equation} \\


$\bm{E}_{l2t}$ and $\bm{E}_{r2t}$ could be formualted as:

\begin{equation}
\bm{E}_{l2t} = \begin{bmatrix}
\bm{R}_{l2t}^{(3\times 3)} & \bm{T}_{l2t}^{(3\times 1)} \\
\bm{0}_{1\times 3} & \bm{1}_{1 \times 1}
\end{bmatrix},
\bm{E}_{r2t} = \begin{bmatrix}
\bm{R}_{r2t}^{(3\times 3)} & \bm{T}_{r2t}^{(3\times 1)} \\
\bm{0}_{1\times 3} & \bm{1}_{1 \times 1}
\end{bmatrix}
\end{equation} \\

$\bm{T}_{l2t}^{(3\times 1)}$ is the coordinate of the left lidar's origin in top-center lidar's frame.  \\
$\bm{T}_{r2t}^{(3\times 1)}$ is the coordinate of the right lidar's origin in top-center lidar's frame. 



\section{Extrinsic Matrix to Euler Angles}
Use tool lidar2lidar calibration tool (\url{https://github.com/gaowexu/SensorsCalibration})
 to manually calibrate point cloud frames captured at the same timestamp and then could obtain the typical extrinsic matrix as below: \\


\begin{equation}
\bm{E}_{l2t} = \begin{bmatrix}
-0.184349 & -0.98286 & 0.0 & 0.54602 \\
0.98286 & -0.184349 & 0.0 & 0.90732 \\
0.0 & 0.0 & 1.0 & -0.642478 \\
0.0 & 0.0 & 0.0 & 1.0 
\end{bmatrix}
\end{equation} \\


\begin{equation}
\bm{E}_{r2t} = \begin{bmatrix}
-0.228178 & 0.972368 & 0.0 & 0.636313 \\
-0.972088 & -0.230317 & 0.0 & -0.78358 \\
0.0 & 0.0 & 1.0 & -0.642478 \\
0.0 & 0.0 & 0.0 & 1.0 
\end{bmatrix}
\end{equation} \\

When the extrinsic parameter matrixs are obtained, then we could convert it to yaw, pitch, roll based on the code below:


\begin{verbatim}
import math

yaw = math.atan2(rotation_matrix[1][0], rotation_matrix[0][0])
pitch = math.atan2(-rotation_matrix[2][0], \
        np.sqrt(rotation_matrix[2][1] ** 2 + \
        rotation_matrix[2][2] ** 2))
roll = math.atan2(rotation_matrix[2][1], rotation_matrix[2][2])
\end{verbatim}


\section{ROS2 Bag Play}
Run the two commands below to associate the left, right frames with front frame.
\begin{verbatim}
ros2 run tf2_ros static_transform_publisher \
0.54602 0.90732 -0.642478 1.7562059520717546 -0.0 0.0 front left
\end{verbatim}

\begin{verbatim}
ros2 run tf2_ros static_transform_publisher \
0.636313 -0.78358 -0.642478 -1.801352191383159 -0.0 0.0 front right
\end{verbatim}

Then the points cloud in three lidars could be visualized at the same time in top-center lidar's coordinate system in rviz2.

\end{document}